\documentclass[12pt]{article}
\usepackage[utf8]{inputenc}
\usepackage[slovene]{babel}

\usepackage{hyperref}
\usepackage{listings} % za naslovnico
\usepackage{amsthm}
\usepackage{amsmath, amssymb, amsfonts}
\usepackage{graphicx}


%\graphicspath{ {./Slike/} }
\usepackage{subcaption} % za side-by-side slike
\usepackage{caption}
\usepackage[
top    = 2.cm,
bottom = 2.cm,
left   = 2.cm,
right  = 2.cm]{geometry}

\usepackage{footnote}
\usepackage{hyperref}
\hypersetup{
    colorlinks=true,
    linkcolor=blue,
    filecolor=magenta,      
    urlcolor=cyan
    }

\makesavenoteenv{tabular}
\title{Domača naloga - 1.del \\
\large Odkrivanje enačb in uporaba predznanja}

\begin{document}
    
\author{Vito Rozman}
\date{\today}
\maketitle



\section{Soočanje s problemom}

Pri izbiri orodaj sem najprej preizkusil linearno regresijo, lasso regresijo, 
rige regresija in nazadnje še modele za iskane enačbe s pomočjo 
\href{https://github.com/MilesCranmer/PySR}{PySR} orodij.
Pri evaluaciji rezultata sem upošteval dve stavri, napako MSE in kopleksnost
enačbe, ki sem jo ocenil z metodo "ostrega pogleda". Pri regresijah sem preizkusil
različne parametre (meja in lambda parameter) in različno generirane spremenljivke. 
Prav tako sem sprobal različne stopnje polinomov.
Postopek iskanaj enačbe sem izvedel v naslednji korakih.
\begin{itemize}
    \item Prvo sem zagnam različne tipe regresij, ki smo 
    jih obravnavali na vajah, brez da bi kakorkoli spremijal spremenljivke.
    Ta pristop se je bil obupen, saj je bila napaka zelo velika.
    \item Druga modifikacija je bila, da sem ustvaril nove spremenljivke 
    gelde na podano predznanje. Generiral sem:
    $T_w - T_a$, $\sqrt{T_w - T_a}$, $\sin(\theta)$,  $\cos(\theta)$, 
    $\frac{1}{\eta}$ in $-\eta$.
    Potem sem ponpvil postopek kot pri 
    prejšni točki in dobil bolše rezultate. Najboljša napaka je znašala $0.029\%$,
    vendar je kopleksnost ostala enako salba.
    \item Tretji pristop je bil z drugimi metodami kot sta algoritem \texttt{BACON} 
    in metoda iz knjižnice \texttt{PySR}. \texttt{BACON} se ni izkazal dobro, \texttt{PySR}
    pa presenetljivo dobro, saj sem dobil nizko napako in dokaj preprosto enačbo.
\end{itemize}


\section{Model za iskanje enačbe}
Kot že omenjeno sem upirabil orodje \texttt{PySR} in sicer funkcijo 
\texttt{PySRRegressor}.

\begin{center}
    \captionof{table}{Nastavitev orodja}
    %\caption{Nastavitve orodja}
    \begin{tabular}{c|c}
        \hline
        Binanrne operacije & $+, -, *, /$ \\
        Druge operacije & $\cos(\cdot),
        \sin(\cdot),
        inv(\cdot),
        \cdot^2 ,
        \sqrt{\cdot} ,
        \cdot^3 ,
        \cos^2(\cdot),
        \sin^2(\cdot)$\\
        Parameter kompleksnosti & $20$ \\
        Največja globina gnezdenja & $10$ \\
        Napaka & MSE \\
        \hline
    \end{tabular}
\end{center}

\section{Rezultat}

Izbranana enačba je oblike:
$$
\sin(\theta)\cdot \left(\frac{3}{2} - \eta\right) \cdot \sqrt{\frac{4(T_w - T_a)}{69}}.
$$
Z dano enačbo sem dobil napako $0.0133\%$, kopleksnot modela pa $13$. Dobljeni enačbi bi zaupal
v približno $95\%$ primerih, saj so dobljene konstante rezultat prileganje vhodnim podatkom. Ostale 
komponente se mi zdijo dokaj vredu, ker ustrezajo predpostavkam domenskega predznaja.
%13,0.013328496,"(square((Tw - Ta) / 17.75263) * (sin(theta) * (1.5037042 - eta)))"


\section*{Opis datotek}
\begin{itemize}
    \item \textbf{preproces.py} spripta kjer generiram nove spremenljivke,
    \item \textbf{eq\_regression.py} spripta s funkcijami regresije in algoritma BACON,
    \item \textbf{NAL1.ipynb} zvezek iskanja enačbe.
\end{itemize}





\end{document}