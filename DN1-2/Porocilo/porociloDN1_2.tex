\documentclass[12pt]{article}
\usepackage[utf8]{inputenc}
\usepackage[slovene]{babel}

\usepackage{hyperref}
\usepackage{listings} % za naslovnico
\usepackage{amsthm}
\usepackage{amsmath, amssymb, amsfonts}
\usepackage{graphicx}


%\graphicspath{ {./Slike/} }
\usepackage{subcaption} % za side-by-side slike
\usepackage[
top    = 2.cm,
bottom = 2.cm,
left   = 2.cm,
right  = 2.cm]{geometry}

\usepackage{footnote}
\makesavenoteenv{tabular}
\title{Domača naloga - 2.del \\
\large Meta učenje}

\begin{document}
    
\author{Vito Rozman}
\date{\today}
\maketitle



\section{Izbira podatkovij}

Pri iskanju najboljšega modela sem najprej pridobil vsa podatkovja iz 
\href{https://www.openml.org/}{OpenML}, za katera obstajajo naloge 
(ang. tasks) za nadzarovano kasifikacijo. Izbral sem podatkovja s število 
primerov med $500-3000$ in s $20-150$ značilk, ker drugače obdržim premalo podatkovij. 
Potem sem odstarnil tista, ki
niso primerna za iskanje podobnosti. Torej da gre za kasifikacijo, vsebujejo samo eno
napovedno spremenljivko, značilke so numeričnega tipa in podatkovja se ne ponavljajo.
Od vseh podatkovij, ki sem jih pridobil sem obdržal $50$ podatkovij.


\section{Meta prostor}

Izbrana podatkovja sem opisal z meta značilkami štirih tipov. To so \emph{splošnega},
\emph{stetističnega}, \emph{info-teoreticnega} in \emph{modelskega} 
(ang. general, statistical, info-theory and model-based). Pri tem sem naletel
na težavo z nedefiniranimi vrednostmi \emph{nan}, kar sem rešil z metodo imputacij. 
Nedefinirane vrednosti sem napolnil s povprečno vrednostjo.

\subsection{Izbira podobnih podatkovij in njihovih modelov}

Za izbiro podobnih podatkovij sem izbral pristop z iskanje k najbližjih sosedov za $k=3$.
Tako sem dobil tri podatkovja PieChart2, cardiotocography in  wdbc. Za vsakega od njih
sem pridobil njegove ocene (ang. evaluations) in posledično model z najboljšim izidom metrike AUC. 
Ker je za vsako podatkovje več taskov sem tako dobil več kot tri modele. 
Ker so nekatere  modeli implementacije iz drugih programskih jezikov, sem 
za nekatere uporablil \emph{sklearn} verzijo ali pa pa jih izločil.
Njaboljši modeli so \textbf{Gaussian Naive Bayes}, \textbf{Decision Tree},
\textbf{Adaptive Boosting}(\emph{base-estimator=DecisionTreeClassifier}) in \textbf{XGBoost}.

\section{Zmogljivost modelov}

Najprej sem razdelil podatke na učne in testne, na učnih sem model učil na testnih pa preveril
njegovo zmogljivost z metriko AUC (ploščina pod ROC krivuljo).
Opisan potopek sem najprej izvedel na neskaliranih podatkih, potem pa še 
na skaliranih ter primerjal rezultate. Iskazalo se je, da so 
skalirani podatki bolje obnesli, vendar pri izbiri najboljšega modela 
niso privedli do večjih razlik.


\begin{center}
    \begin{tabular}{||c| c c||} 
        \hline
        Modeli & AUC - cross validation &  AUC test set \\ [0.5ex] 
        \hline\hline
        Gaussian Naive Bayes & 0.848005  & 0.740600 \\
        \hline
        Decision Tree & 0.753321  & 0.739688\\
        \hline
        Adaptive Boosting & 0.798328 & 0.715982\\
        \hline
        XGBoost & 0.926251 & 0.813512 \\
        \hline
    \end{tabular}
\end{center}
Hiperparametre modelov ki sem jih dobil nisem optimiziral, morda bi bilo smiselno izvesti 
še ta kotak za odločitev najboljšega modela iz meta učenja.
Kot končni model sem izbral model z najboljšim izidom AUC na testni podatakih, to je bil
\textbf{XGBoost} z AUC na testni množici $0.813512$.

\section{Zaključek}

\begin{center}
    \begin{tabular}{||c| c c||} 
        \hline
         & AUC - cross validation &  AUC test set \\ [0.5ex] 
        \hline\hline
        Random Forest - ročno & 0.928039  & 0.887989 \\
        \hline
        Random Forest - avtomatizirano & 0.928101 & 0.847335\\
        \hline
        XGBoost - meta učenje & 0.926251 & 0.813512 \\
        \hline
    \end{tabular}
\end{center}

Pristop z meta učenjem je bil dokaj učinkovit, saj je zmogljivost izbranega modela 
primerljiva ročnemu iskanju in avtomatiziranem iskanju. Ob ponovnem iskanju najboljšega
modela, bi najprej uporabil meta učenje tako bi dobil okvirno modele, ki so primerni za moje 
podatke, nato bi za nekaj izbranih kandidatov izvedel avtomatizirano izbiro hiperparametrov.
Po dani konfiguraciji bi še ročno preveril okolico nastavitve parametrov.

\end{document}